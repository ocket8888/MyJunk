\documentclass[12pt]{article}

\author{Brennan Fieck}
\title{The Good the Bad and the Ugly of Science Communication}
\date{\today}

\begin{document}
\maketitle

Science communication is an inherently difficult field of journalism, because at its core the goal is to explain concepts to an audience that is ill-equipped to understand them. When surmounting this challenge, there are a plethora of "do"s and "don't"s that should be heeded to ensure the best possible chance of imparting knowledge on the reader.\\
\indent A good attempt at science communication will include an array of reliable sources from qualified individuals, avoid absolutism or opinionated interpretations, reflect the writer's honest attempt to understand the source material, eschew sensationalism, and maintain a level of technical jargon comprehensible to the intended audience.\\
\indent In contrast, a poor attempt at science communication makes unfounded claims, endeavors to "steer" the audience's viewpoint - most reprehensibly as a marketing ploy - in some fashion, portrays opinion or unfounded statements as undeniable fact, and ultimately fails to convey any meaningful information. At its most harmless, poor science communication dilutes the information available to general public and erodes trust in the organizations and/or individuals involved. At worst, it can be an attempt to manipulate, misinform, and mislead the public while obscuring real information in an attempt to control the audience for some purpose.\\
\indent To avoid this, it is important that any article or story uphold certain standards. Firstly, it is very important that any claims made be substantiated by a quote from or reference to a person or group of people qualified to speak on the matter. In broad terms, a "reliable source" is an "expert" in the field discussed by the article. This can mean a Master's degree or PhD or several years of experience; often the qualifications are dependent on the subject matter. Furthermore, it is important to include multiple distinct sources, so that the reader can be assured that the entire work isn't based on the ravings of a madman or fringe group. It is also highly beneficial to report conflicting viewpoints, when possible, as it can bolster trust in the article as an unbiased source willing to present all existing information.\\
\indent Also of importance is that the writer not make absolute statements or come to solid conclusions from formative works. Often scientific studies arrive at a conclusion of "it seems to be the case that X" or "this evidence supports theory Y," and the science journalist must resist the temptation to translate these tentative statements into confident claims such as "this proves Z". The difference may appear subtle, but this can be extremely damaging as seen in the recent "vaccines cause autism" scare.\\
\indent It can also be tempting to skew findings in favor of a viewpoint - be it scientific or not - because people have opinions, and journalists are people. It is not only ethically responsible, but effective in building the reader's trust to make the effort to leave personal opinions out of a discussion of raw fact. It should be a journalist's top priority to \emph{inform} the reader, and this means not only keeping opinion-based statements to a minimum, but also making a genuine effort to inform as opposed to advertise. The general public is growing increasingly weary and wary of "clickbait" and "fake news", so titles such as "The Most Amazing Thing Forever !!!11!!ONE!!1" aren't doing a writer any favors, regardless of the work's actual contents.\\
\indent Of course, a writer's attempt to convey information can be totally undermined by an inability on their part to understand the information as it was presented to them. A writer must make a genuine effort to understand the material themselves if they hope to impart any of it on another person. On the other hand, sometimes a writer can go over the heads of the audience when they understand the source material too well. It's important not to confuse the reader by presenting them with new ideas built on a foundation of knowledge they lack. If a brief, broad explanation of background information is insufficient to prepare the audience for the remainder of the discussion, perhaps it needs to be restated in a simpler way or presented to a different audience with more experience in the field.\\
\indent Presenting complex ideas to an audience not necessarily versed in scientific rhetoric is a difficult task, to say the least. It behooves the author of such works, then, to conduct themselves in a manner conducive to garnering trust, relaying real information (i.e. facts not opinions/marketing), and general comprehensibility for the target audience. Many institutions/publications/organizations are working to present and maintain a code of conduct regarding scientific journalism, and it is their hope that these will enable effective, unbiased communication for years to come.

\end{document}