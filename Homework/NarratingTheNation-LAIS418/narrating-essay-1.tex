\documentclass[a4paper,12pt]{article}

%% Language and font encodings
\usepackage[english]{babel}
\usepackage[utf8x]{inputenc}
\usepackage[T1]{fontenc}

%% Sets page size and margins
\usepackage[a4paper,top=1in,bottom=1in,left=1in,right=1in,marginparwidth=1.75cm]{geometry}

%% Useful packages
\usepackage{amsmath}
\usepackage{graphicx}
\usepackage[colorinlistoftodos]{todonotes}
\usepackage[colorlinks=true, allcolors=blue]{hyperref}

\providecommand{\mhfz}[1]{(Mahfouz #1)}

\title{Narrating essay 1}
\author{You}
\linespread{1.7}
\begin{document}
  \begin{titlepage}
    \vspace*{\fill}
    \begin{center}
      {\Huge Essay \#1}\\
      {\LARGE \emph{Divergent Paths}}\\[0.4cm]
      {\Large Brennan W. Fieck}\\
      {\large LAIS 418, A - Narrating the Nation}\\
      \today
    \end{center}
    \vspace*{\fill}
  \end{titlepage}
  
The characters of \textit{Sugar Street}, despite spending their lives in such
close proximity, represent a surprisingly wide array of ideals, creeds, careers
and associations. Nowhere is this dispersion from a common origin more apparent
than in the disparate paths in life taken by Kamal and Ahmad Shawkat. Both of
them pursued education in the College of Arts, and both of them grew up in the
same family which required of each a steadfast defense of the merits of such
instruction. The two men shared a love of literature, as well as a lack of the
religious conviction which seemed omnipresent in the daily lives of most other family
members. What is perhaps most interesting is that certain, impactful events which occurred in
Kamal's formative years were echoed to a degree in Ahmad's youth, further
linking the platforms from which their outlooks on life sprang. However,
despite sharing these common circumstances, the philosophies of the two quickly
diverged. Each different choice or refusal to choose served to magnify all those that preceded
it, the effects cascaded and in time they came to extremely different
conclusions on all manner of subjects, from love to life itself.\\
\indent The first comparison made between Ahmad and Kamal is that of a shared
intellectual pursuit. When assaulted by criticisms of his decision to enroll in
the College of Arts as his uncle had done years before, Ahmed turned to Kamal
for both assistance and sympathy. In fact, so strong were the parallels being
drawn between the two situations that Kamal felt personally slighted by
Ibrahim's comment, "Ahmad, think how Law School has allowed al-Hamzawi's son to
become an important government attorney" \mhfz{26}. However, it is vital to
notice that even this founding similarity conceals dissimilarity. Kamal was
captivated by literature and learning as a result of an infatuation with truth.
Ahmad's somewhat more concrete goal was what he referred to as "intellectual
leadership," or "guiding thought"\mhfz{25}. By which he of course meant using
journalism as a platform for promoting his economic and social ideologies. So
already Ahmed had displayed a trait that his uncle did not possess: a purpose.
The significance of this point became steadily more apparent as time wore on.\\
\indent Even before his entrance into the University, Ahmed's religious views
were inscrutable at best. At the beginning of the story, he was merely evasive.
The first examples of his secularism took the form of mere jokes meant to
provoke his brother. However, he gradually became more disillusioned with the
the religion of his ancestors, owing at least in part to the magazine to which
he subscribed and for which he eventually wrote: \textit{The New Man}. In that
magazine he found a semblance of companionship, sorely needed by the boy who was
vilified by his brother at home who would shout at him "hush, atheist!" and
"Enemy of God!". While attending the University, he grew more bold, openly
mocking his brothers piousness in the midst of his peers. By the time of his
arrest, he had remarked that he and his wife intended to "...live according to
the Marxist faith."\mhfz{270} rather than any religion, and referred to the
latter as "a cultural artifact" and its promise of an afterlife likewise as
"a distracting opiate"\mhfz{297}.
\\
\indent In contrast, Kamal regarded religion with the same uncertainty as
everything
else, from the opening until the closing passages of the novel. Even when first
questioned regarding his beliefs by Riyad Qaldas, who he could identify
as a fellow philosopher (thus not one to pass judgment on philosophy), he attempted to dismiss the inquisition with a shrug.
In the wake of A\"ida's death, he wondered if she found herself in the company
of other loved ones he had lost, yet earlier when his own father passed he still
deflected his mother's attempts to instill faith in him. It seemed that any
given circumstance was as likely to push him further from belief as it was to
pull him closer, and so he remained stagnant.
Rather than even attempting to form any conviction as to the existence of God,
or lack thereof, he remained paralyzed by the fear and doubt that ceaselessly
assailed his mind.\\
\indent For much of his life, Kamal attributed his boundless anxiety to a lost
love in his youth. A\"ida, he often mused, had taken with her life's pleasures
when she left Egypt. Specifically, he used the word "spurn" to describe what
she had done to him. He also laments the more recent loss of his brother,
Fahmy, who was a casualty of political protest.\\
\indent Interestingly, Ahmad found himself in somewhat similar circumstances
while studying at the College of Arts. First, he lost his cousin to childbirth,
who was comparable to a sibling in some respects - not the least of which being
that as she was married to his brother, she was his sister-in-law. Ahmad did
not seem to devote much time to grieving, and it certainly did not
change his demeanor. He still poked fun at his brother and found time to make
friends. One of whom he admired from afar for some time, hoping for more
than friendship before even talking to her. After a year of friendship with
the girl, Ahmad confessed his feelings to the girl, who weighed them against
her lineage, beauty and education and found that the transaction did not not
suit her. Such a cold, calculating approach to a matter of the heart angered
Ahmad greatly, but in time it merely served as a cautionary tale not to mistake
friendship for love when he met his future wife, Sawsan Hammad. The experience
certainly didn't taint the entire notion of a monogamous relationship in his
mind for decades. In fairness, Kamal had known his love much longer, but he was
also much younger, and had had much more time to recover.\\
\indent In any case, it's clear that these two responded to similar situations
in increasingly different ways. Of course, all of human development can not be
summed up to nurture; there is also nature to consider. One might reasonably
argue that there was some difference inherent in the structure of their brains
from birth that allowed Ahmad to deal with his emotions in a manner unavailable
to Kamal. Consider, though: on page 24 Ahmad knows what he wants to do with his
life, and that's what he's doing on page 315 when he is arrested, whereas
Kamal's only certainty is uncertainty until the words of Ahmad himself change
his mind. Ahmad, from an early age, had decided to live according to his moral
values and so when disaster struck he had a path to which he could return.
Comparatively, Kamal professed to believe in nothing and thus nothing was there
to comfort him when he needed comfort. Nothing made him happy because he valued
nothing, save perhaps the perfect theory of a woman he used to know - not even
the woman herself, only the unattainable, perfect idea of a woman held out of
his reach. Ahmad valued his fellow man and his methods of ensuring equality. So
even in the dark and damp of a prison more real than any that had ever held his
uncle, he was able to smile at the companionship of drunks and thieves. Kamal's
mistake, more than anything else, was trying to ascribe some grand meaning to
the mere act of existence. Ahmad's words to him before departing, though,
caused him to think that perhaps a life governed by uncertainty was merely an
attempt at "evasion of responsibility."\mhfz{330}

\begin{quote}
"The duty common to all human beings is perpetual revolution, and that is nothing other than an unceasing effort to further the will of life represented by its progress toward the ideal." -Ahmad Shawkat \mhfz{328}
\end{quote}
  
Perhaps even more important than finding for himself a reason to live, these
words caused Kamal to re-examine the value of his existence in terms of what it
was worth to those around him. "...ask yourself how much longer you will
continue wasting your life."\mhfz{326} he thinks as the mother, who for so long
cherished the life he had considered worthless, lay dying.\\
\indent The fact that Kamal was able to contemplate restructuring his life
after so long indicates that in fact it was not merely in his nature to falter
in the face of adversity while his nephew persevered. If that were the case,
his resolution at the time of his mother's death and the incarceration of two
family members would either be impossible, or a doomed endeavor. The best
explanation, then is that this particular facet of \textit{Sugar Street} serves
not as a mere description of two different men, but as a prescription for the
most general basis on which any life worth living must be built: care about
something. \\
  
\newpage

\nocite{*}
\bibliographystyle{plain}
\bibliography{sample}

\end{document}