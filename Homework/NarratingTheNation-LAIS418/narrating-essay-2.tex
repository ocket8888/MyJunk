%%%%%%%%%%%%%%%%%%%%%%%%%%%%%%%%%%%%%%%%%%%%%%%%%%%%%%%%%%%%%%%%%%%%%
% LaTeX Template: Project Titlepage Modified (v 0.1) by rcx
%
% Original Source: http://www.howtotex.com
% Date: February 2014
% 
% This is a title page template which be used for articles & reports.
% 
% This is the modified version of the original Latex template from
% aforementioned website.
% 
%%%%%%%%%%%%%%%%%%%%%%%%%%%%%%%%%%%%%%%%%%%%%%%%%%%%%%%%%%%%%%%%%%%%%%

\documentclass[12pt]{report}
\usepackage[a4paper]{geometry}
\usepackage[myheadings]{fullpage}
\usepackage{fancyhdr}
\usepackage{lastpage}
\usepackage{graphicx, wrapfig, subcaption, setspace, booktabs}
\usepackage[T1]{fontenc}
\usepackage[font=small, labelfont=bf]{caption}
\usepackage{fourier}
\usepackage[protrusion=true, expansion=true]{microtype}
\usepackage[english]{babel}
\usepackage{sectsty}
\usepackage{url, lipsum}


\newcommand{\HRule}[1]{\rule{\linewidth}{#1}}
\onehalfspacing
\setcounter{tocdepth}{5}
\setcounter{secnumdepth}{5}

%-------------------------------------------------------------------------------
% HEADER & FOOTER
%-------------------------------------------------------------------------------
\pagestyle{fancy}
\fancyhf{}
\setlength\headheight{15pt}
\fancyhead[L]{CWID: 10615596}
\fancyhead[R]{Colorado School of Mines}
\fancyfoot[R]{Page \thepage\ of \pageref{LastPage}}
%-------------------------------------------------------------------------------
% TITLE PAGE
%-------------------------------------------------------------------------------

\begin{document}

\title{
		~\\ [2.0cm]
		\HRule{0.5pt} \\
		\LARGE \textbf{\uppercase{Peru: An Imagined Community}}
		\HRule{2pt} \\ [0.5cm]
		\normalsize Essay \#2 \vspace*{5\baselineskip}}

\date{\today}

\author{
		Brennan W. Fieck \\ 
		Narrating the Nation \\
		LAIS418 - Dr. Straker, James D. }

\maketitle
\newpage

\doublespacing
\indent Anderson and Llosa have fairly similar ideas about what holds a
nation's identity together. Anderson contests that a community is formed
from shared concerns and values, as well as methods of communication and
acting upon those concerns and values. Likewise, Llosa's description of
Peru speaks of media holding popular interest throughout the nation (and
occasionally beyond), but he also discusses subcultures within the nation
that fragment it further, in some respects in direct contradiction with
Anderson's description of emergent states.
\indent At their most basic level, any society is defined primarily by a
distinction that can be drawn between its constituents and those that lay
outside its bounds. Simply put: a society may be loosely understood as a
concept of "us" and "them". Anderson outlines several mechanisms that
served to sequester the proto-South American communities from one another
and the world at large, as well as bind them internally into a national
identity.\\
\indent Firstly, Spain's holdings in the Americas became disjoint from
her over the course of roughly three hundred years - funnily enough
effected largely by Spain's own policies. Intriguingly, many of the modern
South American nations overlay administrative divisions of the old Spanish
American Empire\cite{bib:commie}. The boundaries thereof were originally
somewhat arbitrary, but soon changed shape to match natural barriers,
political impetus, and economic practicality. This had the effect of
constraining the political career of those with such aspirations. They
could change postings and positions, but remained largely encapsulated;
unable to "climb the corporate ladder".\\
\indent Furthermore, in addition to the natural barrier posed by the
Atlantic Ocean, other obstacles barred upward mobility via relocation to
Spain. Without going into too much
unrelated detail, it is important to note that with rapid European
technological advances and territorial expansion came a certain, pervading
attitude toward the indigenous peoples of settled lands. In particular, it
was commonly held that the environment (ecological, cultural, etc.) one
was born in bore monumental, definitive impact into one's character and
ability - both physical and mental. So the mere act of being born on the
uncivilized side of the Atlantic forever marked a
\emph{creole}\footnote{"A person of (at least theoretically) pure European
descent but born in the Americas (and, by later extension, anywhere
outside of Europe)."\cite{bib:commie}} as innately distinct from (inferior
to, in many minds) "true Europeans".\\
\indent Additionally, the period of time
leading up to the eventual revolt of the American colonies, and indeed for
some years afterward were tumultuous for Europe to say the least. Possibly
because of this, Spain undertook aggressive trade regulation policies in
the Americas that considerably favored domestic coffers. Specifically,
outside of small, local regions, intra-continental trade was prohibited.
This facilitated the division of Central and South America into these
"market zones"\cite{bib:commie} where goods did not need to travel all the
way to Spain on their way from producer to consumer. The people living
within these "zones" had a natural need of the same general information
about daily affairs, being grouped as they were by shared commerce, law, and
climate. With the advent of newspaper as a vehicle for this information,
the communities of South America were knit together by common interest and
common intrigue. Perhaps most interestingly, these communities were formed
by an amalgam of conquerer and conquered, with creoles identifying the previously-subjugated and even enslaved indigenous people as fellow countrymen, rather than feeling any connection to Spain.\cite{bib:commie}\\
\indent In Peru, as described by Mario Vargas Llosa, the novellas serve a
similar purpose to that of the newspapers Anderson discussed. The chief
difference being that the novellas conveyed information that was
entertaining rather than strictly pertinent. Like the newspapers, the novellas were widely distributed and consumed by the populace, at one point even drawing the eye of a foreign government. However, it also wouldn't do to
consider the 1970's Central and South American nations disconnected in the
same ways as they were in the early 1800's. Scripts for the novellas were
imported from Cuba\cite{bib:main}- and writers from Bolivia - which
indicates a degree of globalization. This is offset, of course, by all the
work put in by studios to "translate" common Cuban sayings and references to
their rough, Peruvian equivalents. There is also a notion of these other
nations being mutually incompatible, on some level. There is a quality to
Aunt Julia's humor that Mario describes as specifically
"Bolivian"\cite{bib:main}. He also mentions briefly a stereotypical view of
Cuba as a tourist trap and crime den (\emph{a la} Hollywood). All of this is
to say nothing of Pedro's lengthy and colorful tirades against the entirety
of Argentinean existence, which will not be repeated here. It's seems the
proverbial window for a single South American state has closed by the time
Vargos's novel is set (Anderson partially blames insufficient technology to
maintain effective lines of communication for this) and they now see each other as
disparate communities; the manner in which they once saw Spain.\\
\indent Internally to Peru, Mario seems to experience a world that is
connected across narrow gaps rent by class and race. In Lima itself there is a startling difference between the
restaurants, coffee shops and movie theaters frequented by him and his
friends and the abject poverty researched and detailed in novella form by
Pedro for some of his plots. He paints imagery of "grown men and women...
paw[ing] through piles of filth" in their efforts to procure food. It can
be difficult to reconcile this with the capital of a nation, particularly a
capital with the aforementioned amenities held commonplace. Of course, this
is a fictional artwork embedded within a fictional artwork, so some
embellishment could reasonably be expected - despite the novel's heavy basis
in Mario Vargas Llosa's personal life.\\
\indent Peru in the '70s also appeared to have suffered from some ingrained
racist tendencies (not entirely surprising, given the era). In the novel,
a mayor, a mayor's wife, and a (fictional) police officer  find very casual
use for the word "sambo," which today would be considered rather
offensive\footnote{To the best of this white, middle-class, American male's
understanding.}. It also seems that either by choice or necessity, ethnic
minority groups tend to form exclusive communities. This is made most
obvious by the exploration of the tiny villages surrounding Chincha by Mario
and his wedding party. In Chincha itself, it would seem that anyone of
importance to the story was of an un-noteworthy, likely common ethnicity
(presumably Latino), but in the smaller outlying villages the wedding party
only encountered people who were black or mestizos. One mayor commented that 
it was highly unusual for a "white couple [to] come to get married in this
godforsaken village"\cite{bib:main}, further implying a general lack of
association between ethnicities.\\
\indent Both Anderson and Llosa describe a system of mass-media binding
together groups of people, causing them to have a common identity. However,
where Anderson asserted that creoles and natives saw each other as
countrymen - or at least that creoles did more so than they thought of Spain
as their country - in Mario's experience there was much disparity
between the different ethnicities. Of course, this was nowhere near enough
to fragment the nation; neither racial nor class  tensions were violent
(excepting the Shining Path, which was guided by a political agenda). The
groups still shared interest in novellas and information about the world
they lived in.\\
\indent It's important that an anthropologist's analysis of a system
coincide well with the story of someone living in it. \emph{Aunt Julia and
the Scriptwriter} is a work of fiction, but draws significant interest from
the life of a Peruvian. Even if this were not the case, it was set in Peru
and written by a man who experienced Peru, so the novel
ought to reflect in some way what Peru is like. Or at the very least what
someone in Peru thinks of life in Peru, however subjective that may be. In this case, Anderson's description of the importance of communication holds up well, and it was likely not his intention to discuss internal fragmentation in any great detail, so it seems his analysis is plausible. Of course, he did give limitations of technology available to use for communications. He may however, be neglecting other important factors. Once
independence has been won Spain no longer stood as a common foe, and it becomes not unusual at all for a Bolivian man to hate all of Argentina with every fiber of his being. Or for communities to naturally separate due to a sense of racial or socio-economic identity.

\pagebreak
\begin{thebibliography}{99}
  \bibitem{bib:commie} Anderson, Benedict. \emph{Imagined Communities: Reflections on the Origin and Spread of Nationalism.} London: Verso, 1991
  \bibitem{bib:main} Vargas Llosa, Mario. \emph{Aunt Julia and the Scriptwriter.} Editorial Seix Barral, S.A., Spain, 1977
\end{thebibliography}
\end{document}