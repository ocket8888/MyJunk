\documentclass[12pt]{article}
 \usepackage[margin=1in]{geometry} 
\usepackage{amsmath,amsthm,amssymb,amsfonts}
\usepackage{enumitem}
\usepackage{graphicx}
\usepackage{float}
 
\newcommand{\N}{\mathbb{N}}
\newcommand{\Z}{\mathbb{Z}}
 
\newenvironment{problem}[2][Problem]{\begin{trivlist}
\item[\hskip \labelsep {\bfseries #1}\hskip \labelsep {\bfseries #2.}]}{\end{trivlist}}
%If you want to title your bold things something different just make another thing exactly like this but replace "problem" with the name of the thing you want, like theorem or lemma or whatever
 
\begin{document}
 
%\renewcommand{\qedsymbol}{\filledbox}
%Good resources for looking up how to do stuff:
%Binary operators: http://www.access2science.com/latex/Binary.html
%General help: http://en.wikibooks.org/wiki/LaTeX/Mathematics
%Or just google stuff
 
\title{Quantum Computing Assignment 3}
\author{Brennan W. Fieck}
\maketitle

\begin{problem}{3.2.1}
Show that
$$|\psi(t_2)\rangle=e^{-iH(t_2-t_1)/\hbar}|\psi(t_1)\rangle$$
is a solution of the time-dependent Schr\"odinger Equation.
\end{problem}

\begin{proof}[Solution.]~\\
The time-dependent Schr\"odinger Equation in some Hilbert space $\mathcal{H}$ for
some system described by the wave-function $|\Psi\rangle\in\mathcal{H}$ is given by:
$$i\hbar\frac{\delta}{\delta t}|\Psi\rangle=H|\Psi\rangle$$
where $H$ is the Hamiltonian Operator of the Hilbert space in question. If we
choose $t_1$ as a constant while allowing $t_2$ to vary, and measure time as $t=t_2-t_1$ (and also ignore relativistic effects that could cause weird, idiosyncratic notions of time-like basis vectors), we can find the time derivative of $|\psi(t_2)\rangle$ easily enough by inspection.
$$\frac{\delta}{\delta t}|\psi(t_2)\rangle=\frac{-i}{\hbar}He^{-iH(t_2-t_1)/\hbar}|\psi(t_1)\rangle$$

since if $t_1$ is constant, $\delta|\psi(t_1)\rangle/\delta t=0$.
Now simply note that $i\hbar=(-i/\hbar)^{-1}$, and so therefore by trivial algebraic substitution:
$$i\hbar\frac{\delta}{\delta t}|\psi(t_2)\rangle=H|\psi(t_2)\rangle$$
which makes $|\psi(t_2)\rangle$ a solution of the time-dependent Schr\"odinger Equation.
\end{proof}

\begin{problem}{3.4.1(a)}
Prove that if the operators $P_i$ satisfy $P_i^\dagger=P_i$ and $P_i^2=P_i$,
then $P_iP_j=0$ for all $i\neq j$.
\end{problem}

\begin{proof}[Solution.]~\\
Any Hermitian projection operator $P$ in Hilbert space $\mathcal{H}$ can be
written in the form
$$P=|\psi_n\rangle\langle\psi_n|$$
for some $n\in\dim{\mathcal{H}}$ where $|\psi_n\rangle$ are orthonormal, and in
particular part of an orthonormal basis for $\mathcal{H}$. Then for the
operators $P_iP_j$, it follows that 
$$P_iP_j=|\psi_i\rangle\langle\psi_i|\psi_j\rangle\langle\psi_j|=|\psi_i\rangle\delta_{ij}\langle\psi_j|$$
because of the orthonormality restraint on each $|\psi_n\rangle$. Therefore, to
obtain non-zero results from operation of the $P_iP_j$ operator, it is necessary
that $i=j$.
\end{proof}

\begin{problem}{3.4.3}
Verify that a measurement of the Pauli observable $X$ is equivalent to a
complete measurement with respect to the basis $\left\lbrace\frac{1}{\sqrt{2}}(|0\rangle+|1\rangle),\frac{1}{\sqrt{2}}(|0\rangle-|1\rangle)\right\rbrace$
\end{problem}

\begin{proof}[Solution.]~\\
The eigenvalues of $X$ are $1$ and $-1$, corresponding to the normalized eigenvectors
$\frac{1}{\sqrt{2}}(|0\rangle+|1\rangle)$ and $\frac{1}{\sqrt{2}}(|0\rangle-|1\rangle$, respectively. Thus a measurement of it is
equivalent to a complete measurement in the basis spanned by $\frac{1}{\sqrt{2}}(|0\rangle\pm|1\rangle)$
\end{proof}

\begin{problem}{3.5.1}
Find the density matrices of the following states
\begin{enumerate}[label=(\alph*)]
\item $\left\lbrace(|0\rangle,\frac{1}{2}),(|1\rangle,\frac{1}{2})\right\rbrace$
\item $\frac{1}{\sqrt{2}}|0\rangle+\frac{1}{\sqrt{2}}|1\rangle$
\item $\left\lbrace(\frac{1}{\sqrt{2}}|0\rangle+\frac{1}{\sqrt{2}}|1\rangle,\frac{1}{2}),(\frac{1}{\sqrt{2}}|0\rangle-\frac{1}{\sqrt{2}}|1\rangle,\frac{1}{2})\right\rbrace$
\end{enumerate}
\end{problem}

\begin{proof}[Solution.]~\\
\begin{enumerate}[label=(\alph*)]
\item $$\begin{bmatrix}
\frac{1}{2} & \frac{1}{2}\\
\frac{1}{2} & \frac{1}{2}
\end{bmatrix}$$
\item $$\begin{bmatrix}
\frac{1}{2} & \frac{1}{2}\\
\frac{1}{2} & \frac{1}{2}
\end{bmatrix}$$
\item $$\begin{bmatrix}
1 & 0\\
0 & 0
\end{bmatrix}$$
\end{enumerate}
\end{proof}
 
\begin{problem}{C1}
Given is the state vector $|\psi\rangle=\alpha_0|0\rangle+\alpha_1|1\rangle$, $\alpha_0,\alpha_1\in\mathbb{C}$. Find the $(\theta,\psi)$ coordinates of this state on the Bloch sphere. 
\end{problem}
 
\begin{proof}[Solution.]~\\
Most generally, any state vector $|\Psi\rangle$ on the Bloch sphere can be written in the form
$$|\Psi\rangle=\cos\left[\frac{\theta}{2}\right]|0\rangle+\sin\left[\frac{\theta}{2}\right]e^{i\psi}|1\rangle$$
For the given state vector, $\theta$ is easily found as $\theta=2\arccos[\alpha_0]$. The phase factor on the $1$
state is a little uglier. We know the form, and we have an expression for $\theta$, so we can directly see that
$$\alpha_1=\sin\left[\frac{\theta}{2}\right]e^{i\psi}=\sin\left[\frac{2\arccos[\alpha_0]}{2}\right]e^{i\psi}$$
$$=\sqrt{1-\alpha^2}e^{i\psi}\to\frac{\alpha_1}{\sqrt{1-\alpha_0^2}}=e^{i\psi}\to\psi=-i\ln\left[\frac{\alpha_1}{\sqrt{1-\alpha_0^2}}\right]$$
This most likely indicates that the lack of an explicit phase term means that $\psi=0$. So we have the
coordinates
$$(2\arccos[\alpha_0],0)$$
\end{proof}

\begin{problem}{C2}
Let $X=\sigma_x$, $Y=\sigma_y$, and $Z=\sigma_z$ denote the usual Pauli spin matrices. Show that $[Y,Z]=2iX$ and $[Z,X]=2iY$. Recall that the commutator of two operators $A$ and $B$ is given by $[A,B]=AB-BA$.
\end{problem}

\begin{proof}[Solution.]~\\
\begin{itemize}
\item $[Y,Z]=YZ-ZY$
$$=\begin{bmatrix}
0 & -i\\
i & 0
\end{bmatrix}
\begin{bmatrix}
1 & 0\\
0 & -1
\end{bmatrix}-
\begin{bmatrix}
1 & 0\\
0 & -1
\end{bmatrix}
\begin{bmatrix}
0 & -i\\
i & 0
\end{bmatrix}$$
$$=\begin{bmatrix}
0 & i\\
i & 0
\end{bmatrix}-
\begin{bmatrix}
0 & -i\\
-i & 0
\end{bmatrix}$$
$$=\begin{bmatrix}
0 & 2i\\
2i & 0
\end{bmatrix}=2i
\begin{bmatrix}
0 & 1\\
1 & 0
\end{bmatrix}=2iX$$

\item $[Z,X]=ZX-XZ$
$$=\begin{bmatrix}
1 & 0\\
0 & -1
\end{bmatrix}
\begin{bmatrix}
0 & 1\\
1 & 0
\end{bmatrix}-
\begin{bmatrix}
0 & 1\\
1 & 0
\end{bmatrix}
\begin{bmatrix}
1 & 0\\
0 & -1
\end{bmatrix}$$
$$=\begin{bmatrix}
0 & 1\\
-1 & 0
\end{bmatrix}-
\begin{bmatrix}
0 & -1\\
1 & 0
\end{bmatrix}$$
$$=\begin{bmatrix}
0 & 2\\
-2 & 0
\end{bmatrix}=2i
\begin{bmatrix}
0 & -i\\
i & 0
\end{bmatrix}=2iY$$
\end{itemize}
\end{proof}

\begin{problem}{C3}
Show that any 2$\times$2 matrix $A$ can be represented as a linear combination of the Pauli spin matrices $X$, $Y$, and $Z$, and the $(2\times2)$ identity matrix $I$.
\end{problem}

\begin{proof}[Solution.]~\\
The most straightforward proof (though perhaps not simplest) is to analyze entry by entry. Let the matrix $A$ be
represented by
$$\begin{bmatrix}
a & b\\
c & d
\end{bmatrix}$$
Now we must ask if each entry can be formed by linear combinations of the entries of the Pauli matrices and the
2$\times$2 Identity.
\begin{itemize}
\item $a$\\
For any complex number $a$, is it true that it can be decomposed such that $a=\alpha+\beta$. Since the complex
plane's dimension is matched by the free variables here, this is trivially true.
\item $b$\\
For any complex number $b$, is it true that it can be decomposed such that $b=\alpha-i\beta$. Once again, it is
trivial that two free variables will span $\mathbb{C}$
\item $c$\\
For any complex number $c$, is it true that it can be decomposed such that $c=\alpha+i\beta$. Trivially
equivalent to $b$.
\item $d$\\
For any complex number $d$, is it true that it can be decomposed such that $d=\alpha-\beta$. Still two free
variables.
\end{itemize}
So trivially any 2$\times$2 matrix can be composed of linear combinations of the Pauli matrices and the
$2\times2$ Identity.
\end{proof}

\begin{problem}{C4}
Consider a composite system consisting of two qubits. Find the Schmidt decomposition of the states
$$\frac{1}{\sqrt{2}}(|00\rangle+|11\rangle)\textnormal{ and }\frac{1}{\sqrt{3}}(|00\rangle+|01\rangle+|10\rangle).$$
\end{problem}

\begin{proof}[Solution.]~\\
\begin{itemize}
\item $\frac{1}{\sqrt{2}}(|00\rangle+|11\rangle)$\\
Let the Schimdt bases be
$$|\Psi^A_0\rangle=|0\rangle\;\;\;\;|\Psi_1^A\rangle=|1\rangle$$
$$|\Psi^B_0\rangle=|0\rangle\;\;\;\;|\Psi_1^B\rangle=|1\rangle$$
Then this state can be written as 
$$\frac{1}{\sqrt{2}}|\Psi_0^A\rangle\otimes|\Psi_0^B\rangle+\frac{1}{\sqrt{2}}|\Psi_1^A\rangle\otimes|\Psi_1^B\rangle$$
with $p_0=p_1=1/2$.
\item $\frac{1}{\sqrt{3}}(|00\rangle+|01\rangle+|10\rangle)$\\
This one is much less trivial. First construct a nai\"eve matrix form $M$ of the state such that $M_{ij}$ is the sum of the coefficients of state $|ij\rangle$:
$$M=\frac{1}{\sqrt{3}}\begin{bmatrix}
1 & 1\\
1 & 0
\end{bmatrix}$$
Now we can decompose this using the matrix's eigenvectors as:
$$\frac{1}{2}(1+\sqrt{5})|\Psi_0^A\rangle\otimes|\Psi_0^B\rangle+\frac{1}{2}(1-\sqrt{5})|\Psi_1^A\rangle\otimes|\Psi_1^B\rangle$$
where the Schmidt bases are
$$|\Psi^A_0\rangle=\frac{1}{2}(1+\sqrt{5})|0\rangle\;\;\;\;|\Psi_1^A\rangle=\frac{1}{2}(1-\sqrt{5})|0\rangle$$
$$|\Psi^B_0\rangle=|0\rangle\;\;\;\;|\Psi_1^B\rangle=|1\rangle$$
and $p_0=p_1=\frac{1}{2}$
\end{itemize}
\end{proof}
\end{document}