\documentclass[12pt]{article}
 \usepackage[margin=1in]{geometry} 
\usepackage{amsmath,amsthm,amssymb,amsfonts}
\usepackage{enumitem}
\usepackage{graphicx}
\usepackage{float}
\usepackage{qcircuit}
 
\newcommand{\N}{\mathbb{N}}
\newcommand{\Z}{\mathbb{Z}}
 
\newenvironment{problem}[2][Problem]{\begin{trivlist}
\item[\hskip \labelsep {\bfseries #1}\hskip \labelsep {\bfseries #2.}]}{\end{trivlist}}
%If you want to title your bold things something different just make another thing exactly like this but replace "problem" with the name of the thing you want, like theorem or lemma or whatever
 
\begin{document}
 
%\renewcommand{\qedsymbol}{\filledbox}
%Good resources for looking up how to do stuff:
%Binary operators: http://www.access2science.com/latex/Binary.html
%General help: http://en.wikibooks.org/wiki/LaTeX/Mathematics
%Or just google stuff
 
\title{Quantum Computing Assignment 6}
\author{Brennan W. Fieck}
\date{March 1, 2017}
\maketitle

\begin{problem}{4.2.3}~\\
\begin{enumerate}[label=(\alph*)]
\item Prove $XR_y(\theta)X=R_y(-\theta)$ and $XR_z(\theta)X=R_z(-\theta)$.
\item Prove Corollary 4.2.1:\\ \emph{Any 1-qubit gate U can be written in
the form
$$U=e^{i\alpha}AXBXC$$}
\end{enumerate}
\end{problem}

\begin{proof}[Solution.]~\\
\begin{enumerate}[label=(\alph*)]
\item Via direct application:
$$XR_y(\theta)X=\begin{bmatrix}
0 & 1\\
1 & 0
\end{bmatrix}\begin{bmatrix}
\cos\left(\frac{\theta}{2}\right) & -\sin\left(\frac{\theta}{2}\right)\\
\sin\left(\frac{\theta}{2}\right) & \cos\left(\frac{\theta}{2}\right)
\end{bmatrix}\begin{bmatrix}
0 & 1\\
1 & 0
\end{bmatrix}$$ 
$$=\begin{bmatrix}
\sin\left(\frac{\theta}{2}\right) & \cos\left(\frac{\theta}{2}\right)\\
\cos\left(\frac{\theta}{2}\right) & -\sin\left(\frac{\theta}{2}\right)
\end{bmatrix}\begin{bmatrix}
0 & 1\\
1 & 0
\end{bmatrix}=\begin{bmatrix}
\cos\left(\frac{\theta}{2}\right) & \sin\left(\frac{\theta}{2}\right)\\
-\sin\left(\frac{\theta}{2}\right) & \cos\left(\frac{\theta}{2}\right)
\end{bmatrix}$$
Now, since cos is an even function, and sin is an odd function, if we let
$\theta=-\theta$ this becomes
$$\begin{bmatrix}
\cos\left(\frac{-\theta}{2}\right) & \sin\left(\frac{-\theta}{2}\right)\\
-\sin\left(\frac{-\theta}{2}\right) & \cos\left(\frac{-\theta}{2}\right)
\end{bmatrix}=\begin{bmatrix}
\cos\left(\frac{\theta}{2}\right) & -\sin\left(\frac{\theta}{2}\right)\\
\sin\left(\frac{\theta}{2}\right) & \cos\left(\frac{\theta}{2}\right)
\end{bmatrix}$$
$$\to XR_y(\theta)X=R_y(-\theta)$$
and now
$$XR_z(\theta)X=\begin{bmatrix}
0 & 1\\
1 & 0
\end{bmatrix}\begin{bmatrix}
e^{-i\theta} & 0\\
0 & e^{i\theta}
\end{bmatrix}\begin{bmatrix}
0 & 1\\
1 & 0
\end{bmatrix}$$
$$=\begin{bmatrix}
0 & e^{i\theta}\\
e^{-i\theta} & 0
\end{bmatrix}\begin{bmatrix}
0 & 1\\
1 & 0
\end{bmatrix}=\begin{bmatrix}
e^{i\theta} & 0\\
0 & e^{-i\theta}
\end{bmatrix}=R_z(-\theta)$$

\item Well, first of all it may be helpful to figure out what $U$ looks
like.
$$U=e^{i\alpha}\begin{bmatrix}
a_1 & a_2\\
a_3 & a_4
\end{bmatrix}\begin{bmatrix}
0 & 1\\
1 & 0
\end{bmatrix}\begin{bmatrix}
b_1 & b_2\\
b_3 & b_4
\end{bmatrix}\begin{bmatrix}
0 & 1\\
1 & 0
\end{bmatrix}\begin{bmatrix}
c_1 & c_2\\
c_3 & c_4
\end{bmatrix}$$
$$=e^{i\alpha}\begin{bmatrix}
a_1 & a_2\\
a_3 & a_4
\end{bmatrix}\begin{bmatrix}
b_4 & b_3\\
b_2 & b_1
\end{bmatrix}\begin{bmatrix}
c_1 & c_2\\
c_3 & c_4
\end{bmatrix}=e^{i\alpha}\begin{bmatrix}
a_1b_4+a_2b_2 & a_1b_3+a_2b_1\\
a_3b_4+a_4b_2 & a_3b_3+a_4b_1
\end{bmatrix}\begin{bmatrix}
c_1 & c_2\\
c_3 & c_4
\end{bmatrix}$$
$$=e^{i\alpha}\begin{bmatrix}
c_1(a_1b_4+a_2b_2)+c_3(a_1b_3+a_2b_1) & c_2(a_1b_4+a_2b_2)+c_4(a_1b_3+a_2b_1)\\
c_1(a_3b_4+a_4b_2)+c_3(a_3b_3+a_4b_1) &
c_2(a_3b_4+a_4b_2)+c_4(a_3b_3+a_4b_1)
\end{bmatrix}$$
So it may be difficult to see in all that mess, but the gist of it is we
have 4 expressions and 4 totally independent degrees of freedom, which
means that it's possible to define each entry of $U$ independently. This, in turn
implies that any $\mathbb{C}^{2\times2}$ object can be defined.
Furthermore, the leading exponential allows the application of arbitrary
global phase factors, and together these meet the criteria of being able
to express any 1-qubit gate.
\end{enumerate}
\end{proof}

\begin{problem}{4.2.4}
Describe the effect of the CNOT gate with respect to the following bases
\begin{enumerate}[label=(\alph*)]
\item $$B_1=\left\lbrace|0\rangle\left(\frac{|0\rangle+|1\rangle}{\sqrt{2}}\right), |0\rangle\left(\frac{|0\rangle-|1\rangle}{\sqrt{2}}\right), |1\rangle\left(\frac{|0\rangle+|1\rangle}{\sqrt{2}}\right), |1\rangle\left(\frac{|0\rangle-|1\rangle}{\sqrt{2}}\right)\right\rbrace$$
\item \begin{multline}
B_2=\left\lbrace\left(\frac{|0\rangle+|1\rangle}{\sqrt{2}}\right)\left(\frac{|0\rangle+|1\rangle}{\sqrt{2}}\right), \left(\frac{|0\rangle+|1\rangle}{\sqrt{2}}\right)\left(\frac{|0\rangle-|1\rangle}{\sqrt{2}}\right), \left(\frac{|0\rangle-|1\rangle}{\sqrt{2}}\right)\left(\frac{|0\rangle+|1\rangle}{\sqrt{2}}\right),\right. \\ \left.
\left(\frac{|0\rangle-|1\rangle}{\sqrt{2}}\right)\left(\frac{|0\rangle-|1\rangle}{\sqrt{2}}\right)\right\rbrace
\end{multline}
\end{enumerate}
\end{problem}

\begin{proof}[Solution.]~\\
\begin{enumerate}[label=(\alph*)]
\item First, it'll make everything easier to simplify the basis.
$$B_1=\left\lbrace\frac{1}{\sqrt{2}}\left(|00\rangle+|01\rangle\right),\frac{1}{\sqrt{2}}\left(|00\rangle-|01\rangle\right),\frac{1}{\sqrt{2}}\left(|10\rangle+|11\rangle\right),\frac{1}{\sqrt{2}}\left(|10\rangle+|11\rangle\right)\right\rbrace$$
application of the \texttt{CNOT} gate to these states gives
$$\left\lbrace\frac{1}{\sqrt{2}}(|00\rangle+|01\rangle, \frac{1}{\sqrt{2}}\left(|00\rangle-|01\rangle\right),\frac{1}{\sqrt{2}}\left(|11\rangle+|10\rangle\right),\frac{1}{\sqrt{2}}\left(|11\rangle+|10\rangle\right)\right\rbrace$$
which is actually just the same as the original basis, which seems to imply that in this basis, \texttt{CNOT} acts as
an identity.
\item First, it'll make everything easier to simplify the basis.
\begin{multline}
B_2=\left\lbrace\frac{|00\rangle+|01\rangle+|10\rangle+|11\rangle}{2}, \frac{|00\rangle-|01\rangle+|10\rangle-|11\rangle}{2}, \frac{|00\rangle+|01\rangle-|10\rangle-|11\rangle}{2},\right. \\ \left. \frac{|00\rangle-|01\rangle-|10\rangle+|11\rangle}{2}\right\rbrace
\end{multline}
Now we apply the \texttt{CNOT} gate to each component of this basis and see what happens
\begin{multline}
B_2=\left\lbrace\frac{|00\rangle+|01\rangle+|11\rangle+|10\rangle}{2}, \frac{|00\rangle-|01\rangle+|11\rangle-|10\rangle}{2}, \frac{|00\rangle+|01\rangle-|11\rangle-|10\rangle}{2},\right. \\ \left. \frac{|00\rangle-|01\rangle-|11\rangle+|10\rangle}{2}\right\rbrace
\end{multline}
In this basis, the first and third components remain unchanged under a \texttt{CNOT} transformation, and the values
of the second and fourth are exchanged. If we represent this basis symbolically like so
$$B_2=\left\lbrace|A\rangle, |B\rangle, |C\rangle, |D\rangle\right\rbrace$$
then any state $|\psi\rangle$ can be expressed as a linear combination of these basis states
$|\psi\rangle=a|A\rangle+b|B\rangle+c|C\rangle+d|D\rangle$. In this basis, application of the \texttt{CNOT} gate
transforms an arbitrary state $|\psi\rangle$ like this:
$$\texttt{CNOT}|\psi\rangle=a|A\rangle+d|B\rangle+c|C\rangle+b|D\rangle$$
\end{enumerate}
\end{proof}

\begin{problem}{4.2.5}
Prove that the $c$-$U$ gate corresponds to the operator
$$|0\rangle\langle0|\otimes I+|1\rangle\langle1|\otimes U$$
\end{problem}

\begin{proof}[Solution.]~\\
This expression must satisfy the following two conditions:
$$c\text{-}U|0\rangle|\psi\rangle=|0\rangle|\psi\rangle$$
\centering and
$$c\text{-}U|1\rangle|\psi\rangle=|1\rangle U|\psi\rangle$$
\raggedright So let's just try direct application:
$$(|0\rangle\langle0|\otimes I+|1\rangle\langle1|\otimes
U)|0\rangle|\psi\rangle$$
$$=(|0\rangle\langle0|\otimes I)|0\rangle|\psi\rangle+(|1\rangle\langle1|\otimes U)|0\rangle|\psi\rangle$$
$$=(|0\rangle\langle0|0\rangle)\otimes(I|\psi\rangle)+(|1\rangle\langle1|0\rangle)\otimes(U|\psi\rangle)$$
$$=|0\rangle\otimes|\psi\rangle=|0\rangle|\psi\rangle$$
so that works, now the other one:
$$(|0\rangle\langle0|\otimes I+|1\rangle\langle1|\otimes
U)|1\rangle|\psi\rangle$$
$$=(|0\rangle\langle0|\otimes I)|1\rangle|\psi\rangle+(|1\rangle\langle1|\otimes U)|1\rangle|\psi\rangle$$
$$=(|0\rangle\langle0|1\rangle)\otimes(I|\psi\rangle)+(|1\rangle\langle1|1\rangle)\otimes(U|\psi\rangle)$$
$$=|1\rangle\otimes|\psi\rangle=|1\rangle|\psi\rangle$$
so that definition of $c$-$U$ checks out.
\end{proof}

\begin{problem}{4.2.6}
We know that $U$ and $e^{i\theta}U$ are equivalent since they only differ
by a global phase. However, prove that $c$-$U\neq c$-$(e^{i\theta}U)$ for
$\theta$ not equal to an integer multiple of $2\pi$.
\end{problem}

\begin{proof}[Solution.]~\\
From the solution to 4.2.5, the $c$-$(e^{i\theta}U)$ gate would operate
like so:
$$|0\rangle\langle0|\otimes I+|1\rangle\langle1|\otimes e^{i\theta}U$$
Note that the phase term is not applied \emph{globally}, meaning that the
gate is now composed of a combination of two different phases - which are
usually orthogonal. However, if
$\theta$ happens to be an integer multiple of $2\pi$, the phases of those
two terms will match, and in that case $c$-$U=c$-$(e^{i\theta}U)$.
\end{proof}

\begin{problem}{4.2.7}
For a given 1-qubit gate $U$, use the result of Corollary 4.2.1 to
construct a circuit for implementing a $c$-$U$ gate using only CNOT gates,
and single-qubit gates.
\end{problem}

\begin{proof}[Solution.]~\\
Given the results of corollary 4.2.1, any arbitrary gate $U$ can be
deconstructed to the form
$$U=e^{i\alpha}AXBXC$$
So to make a controlled version of $U$, it suffices to make controlled
versions of its constituent gates. The global phase factor will be
ignored. Recall that the controlled version of $X$ is a CNOT gate.
\[
\Qcircuit @C=.5em @R=0em @!R {
& \ctrl{1} & \qw & \push{\rule{.3em}{0em}=\rule{.3em}{0em}} & & \qw & \qw & \ctrl{1} & \qw & \qw & \qw & \ctrl{1} & \qw & \qw & \qw \\
& \gate{U} & \qw & & & \gate{A}&\qw&\targ&\qw&\gate{B}&\qw&\targ&\qw&\gate{C}&\qw
}
\]
Now, to prove that this works, let's find the circuit's transfer function:
$$c\text{-}U=(I\otimes A)(\texttt{CNOT})(I\otimes B)(\texttt{CNOT})(I\otimes C)$$
$$=\begin{bmatrix}
A &\vline& 0\\
\hline
0 & \vline & A
\end{bmatrix}\begin{bmatrix}
I_2 & \vline & 0\\
\hline
0 & \vline & X
\end{bmatrix}\begin{bmatrix}
B &\vline& 0\\
\hline
0 & \vline & B
\end{bmatrix}\begin{bmatrix}
I_2 & \vline & 0\\
\hline
0 & \vline & X
\end{bmatrix}\begin{bmatrix}
C &\vline& 0\\
\hline
0 & \vline & C
\end{bmatrix}$$
$$=\begin{bmatrix}
A & \vline & 0\\
\hline
0 & \vline & AX
\end{bmatrix}\begin{bmatrix}
B & \vline & 0\\
\hline
0 & \vline & BX
\end{bmatrix}\begin{bmatrix}
C &\vline& 0\\
\hline
0 & \vline & C
\end{bmatrix}$$
$$=\begin{bmatrix}
A & \vline & 0\\
\hline
0 & \vline & AX
\end{bmatrix}\begin{bmatrix}
BC & \vline & 0\\
\hline
0 & \vline & BXC
\end{bmatrix}$$
$$=\begin{bmatrix}
ABC & \vline & 0\\
\hline
0 & \vline & AXBXC
\end{bmatrix}=\begin{bmatrix}
I_2 & \vline & 0\\
\hline
0 & \vline & U
\end{bmatrix}$$
\end{proof}

\begin{problem}{F1}
Find the left and right polar decompositions of
the matrix
$$A=\begin{bmatrix}
1 & 2\\
2 & 1
\end{bmatrix}$$
\end{problem}

\begin{proof}[Solution.]~\\
Right and left polar decompositions of a matrix $M$ take the form $UP$ and
$P^\prime U$ respectively, where $P=\sqrt{MM^*}$ and
$P^\prime=\sqrt{M^*M}$. So first it's prudent to find $A^*$:
$$A^*=\begin{bmatrix}
1 & 2\\
2 & 1
\end{bmatrix}=A$$
which is fortunate, because it means that $A^*A=AA^*=A^2$, which means that $P$
and $P^\prime$ must both be exactly $\sqrt{A^2}=A$, and thus the only value $U$
can have is the 2$\times$2 Identity, so the answer is simply:
\begin{itemize}
\item Right
$$A=AI$$
\item Left
$$A=IA$$
\end{itemize}
\end{proof}
 
\end{document}